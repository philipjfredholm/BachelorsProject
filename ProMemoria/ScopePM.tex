\documentclass[a4, 12pt]{article}

\usepackage[margin = 2.5cm]{geometry}
\usepackage{graphicx}
\usepackage{caption}
\usepackage{subcaption}
\usepackage{float}
\usepackage{hyperref}
\setcounter{secnumdepth}{0}

\usepackage{amsmath}
\usepackage{amssymb}
\usepackage{siunitx}
\usepackage{tcolorbox}
\usepackage{xcolor}
\usepackage{listings}
\usepackage{calligra}
\usepackage{mathtools,cases}
\usepackage{empheq}
\DeclareMathAlphabet{\mathcalligra}{T1}{calligra}{m}{n}
\DeclareFontShape{T1}{calligra}{m}{n}{<->s*[2.2]callig15}{}
\newcommand{\scriptr}{\mathcalligra{r}\,}
\newcommand{\boldscriptr}{\pmb{\mathcalligra{r}}\,}






\definecolor{dkgreen}{rgb}{0,0.6,0}
\definecolor{gray}{rgb}{0.5,0.5,0.5}
\definecolor{mauve}{rgb}{0.58,0,0.82}
\lstset{frame=tb,
  language=python,
  aboveskip=3mm,
  belowskip=3mm,
  showstringspaces=false,
  columns=flexible,
  basicstyle={\small\ttfamily},
  numbers=none,
  numberstyle=\tiny\color{gray},
  keywordstyle=\color{blue},
  commentstyle=\color{dkgreen},
  stringstyle=\color{mauve},
  breaklines=true,
  breakatwhitespace=true,
  tabsize=3
}



\title{
Pro Memoria \\
Scope of the Bachelor's Project
}
\author{FYSK03 \\ Philip J. Fredholm}
\date{}


\begin{document}
\maketitle


\section{Note}
These are just random sketches to help me remember things better, the information in here may not be entirely factual nor correct. I have also not proofread it for spell-checking. 
\section{Introduction}
When heavy ions collide, they may form a so called quark-gluon plasma (QGP) which occurs only when a very large number of high energy quarks are able to break free of colour confinement due to space (at least locally) being isotropic in colour (or just white in colour). The QGP is usually form from the intersection of the two colliding heavy nuclei, and its shape is then both determined by the impact parameter and the radii of the nuclei. Since random fluctuations occur, given radii and impact parameters do not uniquely determine the shape of the QGP, but in general it gets a sort of 'almond shape'. It may be shown (maybe through some hydrodynamics that will maybe be covered in FYTA14?) that this leads a pressure gradient which results in an anisotropic expansion, or flow as it is usually called. It has been talked about that QGP is some sort of perfect fluid. This flow is interesting as measuring it is necessary to reconstruct what happened in the QGP. \newline 
\indent To measure the flow, one usually fits the particle count to a 3D Fourier expansion of the shape. The shape is in turn determined by the number of measured particles per a given direction. What is of main interest to this project is the fitting parameter $v_2$. The main difficulty in measuring the flow is that the number of particles in a given direction may not arise solely from effects of the QGP, but instead from effects of quantum chromodynamics (QCD). The main culprit here are the collimated sprays of particles known as jets, which are very correlated due to effects of QCD. In order to get an accurate value of $v_2$, one must account for this non-QGP correlation between particles. \newline	
\indent One common way to do this is to correlate $n$ particles to get a value for $v_2\{n\}$ by introducing an forced separation of the pseudorapidity $n$ for the values we correlate. However, this drastically reduces the number of available particles to correlate and may eventually lead to a running of usually data to do statistics on. To get around this issue, this project will use a different approach where particles in the central time projection chamber (TPC) are correlated with particles detected along the beamline in the forward multiplicity detector (FMD). Using some mathematics and combining 'backwards' and 'forwards' FMD:s, it will be possible to calculate a value for $v_2$ which is only for the TCP. \newline
\indent This still leaves the problem of the correlation between particles in jets. However, the intent is to first to this for proton-proton collisions (pp collisions). This way, we will get no effects of a QGP (since the proton is too small) and only what would otherwise be considered noise. This is useful as QCD (somehow) predicts that these noise effects will be on the same form regardless of the size of the ion. Thus, if the QCD 'background' is known, an event of an i.e lead-lead collision (Pb-Pb collision) may be fitted to the sum of the known background multiplied by a scaling factor and the Fourier expansion of the flow. The aim of this project is to determine the flow of the Pb-Pb collisions by first removing the background by studying p-p collisions.



\end{document}